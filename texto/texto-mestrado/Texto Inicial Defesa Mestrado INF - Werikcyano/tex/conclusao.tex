\chapter{Conclusão}
\label{cap:conclusao}

Este trabalho investigou a integração de Curriculum Learning com Self-play para aprendizado por reforço no contexto do futebol de robôs da categoria SSL-EL. Os resultados demonstraram claramente a superioridade da abordagem combinada em relação às técnicas isoladas, validando a hipótese central da pesquisa.

As principais contribuições deste trabalho incluem:

\begin{enumerate}
    \item \textbf{Metodologia estruturada}: Desenvolvimento de um framework para integração de Curriculum Learning e Self-play, com definição clara de estágios progressivos e critérios de transição adaptativos.
    
    \item \textbf{Evidência empírica}: Comprovação quantitativa dos benefícios da abordagem combinada, com ganhos de 86\% na taxa de vitória em torneios comparativos (430 vitórias em 500 partidas) e aumento impressionante na média de gols por partida (2,024 vs 0,018 do Full Self-play, mais de 100 vezes superior).
    
    \item \textbf{Eficiência computacional}: Redução de aproximadamente 15\% no tempo total de treinamento (7,4 horas versus 8,7 horas), fator crítico para aplicações práticas.
    
    \item \textbf{Estabilidade aprimorada}: Demonstração de um processo de aprendizado mais consistente e menos volátil, evidenciado pelas métricas básicas de RL (entropia da política, perda da política e variância explicada).
\end{enumerate}

Os resultados sugerem que o Curriculum Learning proporciona uma base técnica sólida que potencializa os benefícios do Self-play. Agentes treinados com Curriculum desenvolvem inicialmente habilidades defensivas e de controle básico que, quando refinadas através do Self-play, evoluem para comportamentos táticos sofisticados e eficientes.

\section{Limitações}

As principais limitações identificadas neste estudo incluem:

\begin{itemize}
    \item \textbf{Reality gap}: Todos os experimentos foram realizados em simulação, persistindo o desafio de transferência para robôs físicos.
    
    \item \textbf{Sensibilidade paramétrica}: A eficácia do curriculum depende significativamente do design apropriado das tarefas e critérios de promoção.
    
    \item \textbf{Especificidade do domínio}: Embora os princípios sejam potencialmente generalizáveis, a validação ocorreu especificamente no contexto do futebol de robôs.
\end{itemize}

\section{Trabalhos Futuros}

O potencial demonstrado pela abordagem sugere diversas direções promissoras:

\begin{itemize}
    \item \textbf{Transferência para robôs reais}: Investigação de técnicas para mitigar o reality gap e aplicar o conhecimento adquirido em simulação em ambientes físicos.
    
    \item \textbf{Generalização a outros domínios}: Adaptação da metodologia para diferentes contextos multiagente competitivos e cooperativos.
    
    \item \textbf{Otimização automática de currículos}: Desenvolvimento de algoritmos para geração e adaptação automática de sequências de tarefas, reduzindo a necessidade de design manual.
    
    \item \textbf{Exploração de arquiteturas híbridas}: Integração do framework proposto com técnicas emergentes como aprendizado por demonstração e modelos baseados em memória.
\end{itemize}

Em síntese, este trabalho demonstra que a abordagem estruturada do aprendizado, combinando Curriculum Learning e Self-play, representa uma estratégia eficaz para desenvolver agentes com desempenho superior em ambientes complexos multiagente. A metodologia proposta oferece um caminho promissor para superar limitações atuais em RL e avançar o estado da arte em sistemas autônomos inteligentes.

