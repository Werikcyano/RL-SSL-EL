\chapter{Conclusão}
\label{cap:conclusao}

Este trabalho investigou a aplicação de Curriculum Learning no contexto do futebol de robôs da categoria SSL-EL, com o objetivo de melhorar o processo de aprendizagem em ambientes complexos multiagente. A abordagem proposta envolveu a decomposição de tarefas complexas em subtarefas mais simples, permitindo um aprendizado gradual e progressivo dos agentes. A implementação foi realizada através de modificações estratégicas no framework existente, com a criação de um ambiente adaptativo capaz de suportar diferentes níveis de complexidade no treinamento.

Os resultados preliminares, embora apresentando uma recompensa média ligeiramente inferior nas primeiras 300 steps de treinamento (2,813 vs 3,034 do método default), demonstraram aspectos promissores da abordagem proposta. A análise do tamanho dos episódios revelou um processo de aprendizado mais estável e consistente com o Curriculum Learning, com uma curva de aprendizado mais suave e menos variações bruscas. Esta estabilidade sugere um potencial benefício para o desenvolvimento de políticas mais robustas e generalizáveis no longo prazo.

A implementação do sistema de curriculum através da classe SSLCurriculumEnv e do CurriculumCallback mostrou-se efetiva na gestão da progressão entre níveis de dificuldade. O sistema de recompensas adaptativo e os mecanismos de monitoramento contínuo permitiram uma transição suave entre os diferentes estágios do aprendizado, contribuindo para a formação de uma base sólida de habilidades fundamentais antes da exposição a cenários mais desafiadores.

Como trabalhos futuros, sugere-se a extensão do período de treinamento para avaliar os benefícios do Curriculum Learning em horizontes de tempo mais longos, bem como a investigação de estruturas curriculares mais complexas que possam incorporar aspectos táticos e estratégicos do futebol de robôs. Além disso, seria interessante explorar a integração de técnicas de self-play com o curriculum learning para potencializar ainda mais o processo de aprendizagem em cenários competitivos multiagente.

\section{Limitações}

Uma das principais limitações encontradas neste trabalho foi o tempo relativamente curto de treinamento (300 steps), que pode não ter sido suficiente para demonstrar todo o potencial do Curriculum Learning em cenários mais complexos. Além disso, a implementação atual focou principalmente em aspectos básicos do controle dos robôs, não abordando elementos táticos mais sofisticados do futebol.

Outra limitação significativa foi a simplicidade da estrutura curricular implementada, que poderia ser expandida para incluir uma maior variedade de níveis e desafios progressivos. O sistema atual também não contempla mecanismos para adaptação dinâmica do currículo com base no desempenho individual de cada agente, o que poderia otimizar ainda mais o processo de aprendizagem.

\section{Trabalhos Futuros}

Para trabalhos futuros, diversas direções promissoras podem ser exploradas:

\begin{itemize}
    \item Desenvolvimento de estruturas curriculares mais sofisticadas, incorporando elementos táticos e estratégicos do futebol de robôs
    \item Implementação de mecanismos de adaptação dinâmica do currículo baseados no desempenho individual dos agentes
    \item Integração de técnicas de self-play com o Curriculum Learning para melhorar o aprendizado em cenários competitivos
    \item Investigação de métodos para transferência de conhecimento entre diferentes níveis do currículo
\end{itemize}

