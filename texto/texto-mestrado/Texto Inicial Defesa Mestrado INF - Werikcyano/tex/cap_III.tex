\chapter{Metodologia}
\label{cap:metodologia}

\section{Implementação do Curriculum Learning}

\subsection{Progressão Estruturada do Aprendizado}

A metodologia proposta baseia-se em uma progressão estruturada de aprendizado através de três tarefas principais, com níveis crescentes de complexidade:

\begin{enumerate}
    \item \textbf{Navegação Básica:} O agente deve aprender a se movimentar pelo campo e alcançar pontos específicos, sem obstáculos. Esta tarefa inicial estabelece as bases do controle de movimento.
    
    \item \textbf{Navegação com Obstáculos:} Introduz obstáculos estáticos no ambiente, exigindo que o agente aprenda a planejar rotas e desviar de colisões enquanto mantém o objetivo de alcançar pontos específicos.
    
    \item \textbf{Navegação Dinâmica:} Incorpora elementos móveis e perturbações ambientais, simulando condições mais próximas de uma partida real.
\end{enumerate}

A escolha desta progressão se justifica pela necessidade de construir gradualmente as habilidades necessárias para o jogo, permitindo que o agente consolide competências fundamentais antes de enfrentar cenários mais complexos.

\subsection{Métricas de Avaliação e Progresso}

Para avaliar o desempenho do agente durante o treinamento, implementamos as seguintes métricas:

\begin{itemize}
    \item \textbf{Taxa de Sucesso:} Percentual de conclusão bem-sucedida das tarefas
    \item \textbf{Tempo de Conclusão:} Duração média para completar cada tarefa
    \item \textbf{Eficiência de Trajetória:} Razão entre o caminho ideal e o realizado
    \item \textbf{Robustez:} Capacidade de manter o desempenho sob perturbações
\end{itemize}

O monitoramento destas métricas é realizado continuamente através de um sistema automatizado de coleta e análise de dados.

\section{Flexibilidade e Configuração do Treinamento}

\subsection{Modos de Treinamento}

O sistema implementa dois modos principais de treinamento:

\begin{itemize}
    \item \textbf{Curriculum Learning:} Progressão estruturada de tarefas
    \item \textbf{Self-play:} Treinamento através de confrontos entre versões do agente
\end{itemize}

A integração entre estes modos é gerenciada por um sistema de avaliação que determina quando o agente está pronto para progredir no curriculum ou alternar para self-play.

\subsection{Parametrização do Ambiente}

O ambiente de treinamento possui diversos parâmetros configuráveis:

\begin{itemize}
    \item Dimensões e características do campo
    \item Velocidade de simulação
    \item Intensidade de perturbações
    \item Frequência de amostragem
\end{itemize}

Cada parâmetro influencia diretamente o processo de aprendizagem, permitindo ajustes finos para otimizar o treinamento.

\section{Integração e Compatibilidade}

\subsection{Adaptação ao Sistema Existente}

A implementação mantém compatibilidade com o sistema VSSS através de:

\begin{itemize}
    \item Interface padronizada de comunicação
    \item Preservação do formato de dados existente
    \item Modularização do código novo
\end{itemize}

\subsection{Transição entre Modos de Treinamento}

O sistema gerencia transições entre modos de treinamento através de:

\begin{itemize}
    \item Checkpoints automáticos
    \item Sistema de versionamento de modelos
    \item Métricas de progresso para decisão de transição
\end{itemize}

\section{Monitoramento e Documentação}

\subsection{Sistema de Logs}

O sistema de logs implementado registra:

\begin{itemize}
    \item Métricas de desempenho por episódio
    \item Estados do ambiente e ações do agente
    \item Eventos de transição entre tarefas
    \item Erros e exceções durante o treinamento
\end{itemize}

\subsection{Visualização e Acompanhamento}

O progresso do treinamento é monitorado através de:

\begin{itemize}
    \item Dashboards no Tensorboard
    \item Gráficos de evolução das métricas
    \item Relatórios automáticos de progresso
    \item Documentação contínua das observações
\end{itemize}

A visualização em tempo real permite ajustes rápidos nos parâmetros de treinamento quando necessário.