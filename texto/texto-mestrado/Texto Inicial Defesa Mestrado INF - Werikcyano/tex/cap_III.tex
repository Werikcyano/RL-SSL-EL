\chapter{Metodologia}
\label{cap:metodologia}

\section{Implementação do Curriculum Learning}

\subsection{Inspiração para modelagem do curriculum}

A modelagem do curriculum foi inspirada em casos reais de jogos eletrônicos populares que implementam sistemas de treinamento progressivo. Jogos como FIFA e Rocket League possuem seções dedicadas ao treinamento que seguem uma abordagem gradual de aprendizado, muito similar aos conceitos fundamentais do Curriculum Learning. No FIFA, por exemplo, o jogador começa aprendendo habilidades básicas como passes, chutes e dribles isoladamente, antes de progredir para situações mais complexas de jogo. De forma análoga, no Rocket League, os jogadores são introduzidos primeiro aos controles básicos do carro, como aceleração e saltos, evoluindo gradualmente para manobras aéreas e jogadas táticas elaboradas. Esta progressão natural do aprendizado, onde conceitos fundamentais são dominados antes da exposição a cenários mais desafiadores, alinha-se perfeitamente com os princípios do Curriculum Learning. A técnica propõe justamente esta abordagem estruturada, onde o agente é exposto a tarefas progressivamente mais complexas, permitindo que construa uma base sólida de conhecimento antes de enfrentar situações que exigem a combinação de múltiplas habilidades. Esta inspiração nos levou a modelar nosso curriculum de forma similar, começando com fundamentos básicos do futebol de robôs e progressivamente introduzindo cenários mais desafiadores e complexos.

\subsection{Isolamento e classificação de tarefas}

\subsection{Modularização dos algoritmos de aprendizagem}



\section{Parametrização do Ambiente}

\subsection{Cenários de treinamento}

\subsection{Recompensa}
%lembrar que as recompensas são varáveis de acordo com o nível de dificuldade



\section{Pipeline de Integração e Continuidade do Treinamento}

\subsection{Adaptação ao Sistema Existente}
%adaptar ao contexto do SSL-EL

\subsection{Transição entre Modos de Treinamento}



\section{Métricas e Avaliação}

Para avaliar o desempenho do agente durante o treinamento, implementamos as seguintes métricas:

\begin{itemize}
    \item \textbf{Taxa de Sucesso:} Percentual de conclusão bem-sucedida das tarefas
    \item \textbf{Tempo de Conclusão:} Duração média para completar cada tarefa
    \item \textbf{Eficiência de Trajetória:} Razão entre o caminho ideal e o realizado
    \item \textbf{Robustez:} Capacidade de manter o desempenho sob perturbações
\end{itemize}

O monitoramento destas métricas é realizado continuamente através de um sistema automatizado de coleta e análise de dados.