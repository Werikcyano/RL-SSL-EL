
\chapter{Introdução}
\label{cap:intro}

O Aprendizado por Reforço (Reinforcement Learning - RL) tem se destacado como uma abordagem promissora para o desenvolvimento de sistemas inteligentes capazes de aprender através da interação com o ambiente. Como apresentado no Capítulo \ref{cap:fund}, o RL permite que agentes aprendam políticas de decisão de forma autônoma, maximizando recompensas acumuladas ao longo do tempo através de processos de tentativa e erro.

Um dos principais desafios no RL, especialmente em cenários complexos como o futebol de robôs, é o aprendizado de tarefas que envolvem múltiplos agentes e horizontes de tempo extensos. Como discutido no Capítulo \ref{cap:fund}, a não-estacionariedade do ambiente e a necessidade de coordenação entre agentes tornam o processo de aprendizagem significativamente mais desafiador.

Para endereçar esses desafios, este trabalho propõe a utilização de Curriculum Learning no contexto do futebol de robôs, especificamente na categoria SSL-EL. O Curriculum Learning, detalhado no Capítulo \ref{cap:metodologia}, permite uma abordagem gradual ao aprendizado, onde os agentes são expostos a tarefas progressivamente mais complexas, facilitando o desenvolvimento de habilidades fundamentais antes de enfrentar cenários mais desafiadores.

A implementação, descrita no Capítulo \ref{cap:metodologia}, integra o Curriculum Learning ao framework existente através de modificações estratégicas na arquitetura do sistema. O ambiente foi adaptado para suportar diferentes níveis de complexidade, com um sistema robusto de monitoramento e progressão entre níveis.

Os resultados preliminares, apresentados no Capítulo \ref{cap:resultados}, demonstram o potencial da abordagem proposta. Embora o método tradicional tenha apresentado uma recompensa média ligeiramente superior nas fases iniciais de treinamento (3,034 vs 2,813), o Curriculum Learning mostrou indícios de maior estabilidade e potencial para melhor generalização no longo prazo.

O objetivo principal desta pesquisa é avaliar a efetividade do Curriculum Learning como estratégia para melhorar o aprendizado em ambientes complexos de futebol de robôs. Através da decomposição de tarefas complexas em subtarefas mais simples e da progressão gradual entre níveis de dificuldade, busca-se desenvolver agentes mais robustos e capazes de lidar com os desafios inerentes ao domínio do futebol de robôs.

