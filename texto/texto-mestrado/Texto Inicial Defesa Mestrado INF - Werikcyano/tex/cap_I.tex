
\chapter{Introdução}
\label{cap:intro}

O futebol de robôs representa um domínio desafiador para a aplicação de técnicas de Inteligência Artificial, combinando aspectos complexos de percepção, tomada de decisão e controle em tempo real. Neste contexto, o Aprendizado por Reforço (Reinforcement Learning - RL) emergiu como uma abordagem promissora, permitindo que agentes desenvolvam comportamentos sofisticados através da interação direta com o ambiente \cite{sutton}. No entanto, a complexidade inerente ao domínio do futebol de robôs apresenta desafios significativos para o aprendizado efetivo.

Um dos principais obstáculos no desenvolvimento de agentes para futebol de robôs através de RL é a necessidade de aprender múltiplas habilidades interdependentes simultaneamente. Os agentes precisam dominar desde capacidades básicas, como navegação e controle da bola, até comportamentos táticos complexos que envolvem coordenação multiagente. Esta multiplicidade de habilidades, combinada com a natureza contínua do espaço de estados e ações, torna o processo de aprendizagem particularmente desafiador.

Para endereçar estes desafios, este trabalho propõe a aplicação de Curriculum Learning como estratégia para melhorar a eficiência e eficácia do processo de aprendizagem em futebol de robôs. O Curriculum Learning permite uma abordagem estruturada ao aprendizado, onde os agentes são expostos a tarefas progressivamente mais complexas, facilitando a aquisição gradual de competências fundamentais. Esta abordagem é especialmente relevante no contexto da categoria SSL-EL (Small Size League - Entry Level), onde os agentes precisam desenvolver habilidades básicas antes de enfrentar cenários competitivos completos \cite{regras_ssl_el_2024}.

A motivação principal deste trabalho surge da observação de trabalhos anteriores \cite{bruno_brandao}, que demonstrou a viabilidade do uso de Aprendizado por Reforço no contexto de futebol de robôs através do algoritmo Proximal Policy Optimization (PPO) em uma abordagem multi-agente com política compartilhada. No entanto, aplicar diretamente métodos de RL ao problema completo do futebol de robôs, sem uma estrutura progressiva de aprendizado, frequentemente resulta em um processo ineficiente e instável. Inspirado pela forma como jogos populares como FIFA e Rocket League introduzem novos jogadores através de centros de treinamento antes de permitir a competição direta, este trabalho propõe uma abordagem baseada em Curriculum Learning para estruturar o processo de aprendizagem em etapas progressivas. Esta estratégia permite que os agentes desenvolvam primeiro habilidades fundamentais antes de enfrentarem cenários mais complexos, similar ao processo natural de desenvolvimento de habilidades em jogadores humanos \cite{relay_long_horizon}.

O objetivo geral deste trabalho é desenvolver e avaliar uma metodologia baseada em Curriculum Learning para melhorar o processo de aprendizagem de agentes em futebol de robôs. Especificamente, busca-se:

1. Desenvolver uma estrutura de curriculum que decomponha o aprendizado em estágios progressivos, começando com habilidades fundamentais como chute básico e progredindo até comportamentos mais complexos;

2. Implementar um sistema de transição adaptativo entre níveis do curriculum, garantindo que os agentes desenvolvam uma base sólida antes de progredir para tarefas mais desafiadoras;

3. Avaliar comparativamente o desempenho de agentes treinados com e sem Curriculum Learning, considerando métricas como eficiência no aprendizado e qualidade final do comportamento aprendido.

As principais contribuições esperadas deste trabalho incluem uma metodologia estruturada para aplicação de Curriculum Learning em futebol de robôs, um framework adaptativo para progressão entre níveis de complexidade e evidências empíricas sobre a efetividade do Curriculum Learning em melhorar o processo de aprendizagem em ambientes complexos multiagente. Este trabalho utiliza como base a implementação do trabalho \textit{Multiagent Reinforcement Learning for Strategic Decision Making and Control in Robotic Soccer Through Self-Play} \cite{bruno_brandao} realizada pela equipe Pequi Mecânico \cite{pequi_mecanico}, que por sua vez foi desenvolvida utilizando o framework RSoccer \cite{rSoccer} da equipe RobôCIn \cite{robocin}. O código fonte completo desta solução está disponível em \url{https://github.com/Werikcyano/RL-SSL-EL}.
