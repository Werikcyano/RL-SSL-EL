\chaves{Aprendizado por Reforço, Curriculum Learning, Futebol de Robôs, Multiagente, Sistema de Recompensas, Treinamento Progressivo}

\begin{resumo} 

Este trabalho investiga a aplicação de Curriculum Learning no contexto do futebol de robôs da categoria SSL-EL, com foco no aprimoramento do processo de aprendizagem em ambientes complexos multiagente. A abordagem proposta utiliza o Aprendizado por Reforço como base, implementando um sistema de currículo que decompõe tarefas complexas em subtarefas mais simples para permitir um aprendizado gradual e progressivo dos agentes. A metodologia desenvolvida inclui a adaptação do framework existente através da classe SSLCurriculumEnv, que gerencia diferentes níveis de complexidade no treinamento, e do CurriculumCallback para monitoramento e progressão entre níveis. Os resultados preliminares, obtidos em 300 steps de treinamento, demonstram que embora o método tradicional tenha apresentado uma recompensa média ligeiramente superior (3,034 vs 2,813), o Curriculum Learning mostrou indícios de maior estabilidade e potencial para melhor generalização no longo prazo, especialmente na análise do tamanho dos episódios e na consistência do aprendizado.

\end{resumo}

