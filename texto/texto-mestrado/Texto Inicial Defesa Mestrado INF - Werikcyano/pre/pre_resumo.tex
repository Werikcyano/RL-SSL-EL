\chaves{Aprendizado por Reforço, Curriculum Learning, Futebol de Robôs, Multiagente, Sistema de Recompensas, Treinamento Progressivo}

\begin{resumo} 
Este trabalho investiga a integração de Curriculum Learning com Self-play para aprendizado por reforço no contexto do futebol de robôs da categoria SSL-EL. A pesquisa aborda o desafio do desenvolvimento de políticas eficientes em ambientes complexos multiagente, propondo uma metodologia estruturada que decompõe o aprendizado em estágios progressivos. O framework implementado estabelece critérios adaptativos de transição entre tarefas, permitindo que os agentes desenvolvam inicialmente habilidades fundamentais antes de enfrentarem cenários competitivos completos. Os resultados experimentais demonstram claramente a superioridade da abordagem combinada, com taxa de vitória significativamente maior em torneios competitivos quando comparada ao Full Self-play tradicional, além de expressivo aumento na média de gols por partida. Adicionalmente, observou-se redução substancial no tempo total de treinamento e maior estabilidade no processo de aprendizado, evidenciada por métricas como entropia da política, perda da política e variância explicada. As análises confirmam que o Curriculum Learning proporciona uma base técnica sólida que potencializa os benefícios do Self-play, resultando em agentes com capacidades táticas mais sofisticadas e eficientes.
\end{resumo}

